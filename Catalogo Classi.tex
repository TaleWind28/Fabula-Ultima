\documentclass{article}

\usepackage[italian]{babel}
\usepackage[a4paper, margin=2cm]{geometry}
\usepackage{xcolor}
\usepackage{tocloft}
\usepackage{titlesec}
\definecolor{bluscuro}{RGB}{0, 0, 139}  % Definizione di un blu scuro
\usepackage{amsmath}
\usepackage{graphicx}
\usepackage[colorlinks=true, allcolors=bluscuro]{hyperref}
\usepackage{wrapfig}
\usepackage{caption}

\title{Categorie di Classi}
\author{TaleWind}
\begin{document}
\maketitle
\tableofcontents
\section{Introduzione}
Fabula Ultima offre molte classi al giocatore per poter vivere al meglio la sua esperienza, però spesso sorgono dei dubbi su quali classi siano più adatte per il nostro scopo sia esso il fare Danni, fornire supporto oppure semplicemente diventare un healer a tutti gli effetti.\\
Questo documento punta ad aiutare i player in difficoltà ad aprocciarsi alla creazione del personaggio di Fabula Ultima, Enjoy!

\section{Categorie di classi}
Nonostante esistano 27 classi è possibile, fortunatamente, dividerle in 3 categorie principali:
\begin{itemize}
    \item \textbf{Damage} : Classi il cui scopo principale è infliggere danni ingenti.
    \item \textbf{Support}: Classi il cui scopo principale è proteggere il party riducendo le prestazioni di combattimento nemiche.
    \item \textbf{Utility}: Classi il cui scopo principale è il miglioramento delle prestazioni, individuali e/o del party, dentro e fuori dal combattimento, senza agire esplicitamente sul danno.
\end{itemize}
Queste categorie sono volutamente generiche e pertanto non incapsulano perfettamente il funzionamento di 
tutte le classi, però consentono di effettuare una scrematura inziale in base al ruolo che il nostro personaggio 
avrà all'interno del party.\\
Per comodità andremo a suddividere, almeno per questo momento, le classi in base al manuale di provenienza:
\subsection{Manuale Base}
\begin{itemize}
    \item \textbf{Damage} : Elementalista, Arcanista, Lama Oscura.
    \item \textbf{Support}: Guardiano, Spiritista, Oratore, Sapiente.
    \item \textbf{Utility}: Artefice, Canaglia, Chimerista, Entropista, Furia, Maestro D'armi, Sapiente, Tiratore, Viandante
\end{itemize}

\subsection{Manuale High Fantasy}

\begin{itemize}
    \item \textbf{Damage} : Cantore.
    \item \textbf{Support}: Simbolista.
    \item \textbf{Utility}: Comandante, Danzatore.
\end{itemize}

\subsection{Manuale Tecno Fantasy}

\begin{itemize}
    \item \textbf{Damage} : Mutante.
    \item \textbf{Utility}: Esper, Pilota.
\end{itemize}

\subsection{Manuale Natural Fantasy}

\begin{itemize}
    \item \textbf{Damage} : Invocatore.
    \item \textbf{Support}: Gourmet.
    \item \textbf{Utility}: Floralista, Mercante.
\end{itemize}

Essendo molto banali come criteri è necessario analizzare il playstile di ogni classe.








\end{document}

