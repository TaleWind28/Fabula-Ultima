\documentclass{article}

\usepackage[italian]{babel}
\usepackage[a4paper, margin=2cm]{geometry}
\usepackage{xcolor}
\usepackage{tocloft}
\usepackage{titlesec}
\definecolor{bluscuro}{RGB}{0, 0, 139}  % Definizione di un blu scuro
\usepackage{amsmath}
\usepackage{graphicx}
\usepackage[colorlinks=true, allcolors=bluscuro]{hyperref}
\usepackage{wrapfig}
\usepackage{caption}

% Cambiare il colore delle sezioni
\titleformat{\section}
  {\color{bluscuro}\Large\bfseries} % Stile: colore + grande + grassetto
  {\thesection}{1em}{}

\titleformat{\subsection}
  {\color{bluscuro}\large\bfseries} % Stile: colore + medio + grassetto
  {\thesubsection}{1em}{}

\renewcommand{\cfttoctitlefont}{\color{bluscuro}\huge\bfseries} % Titolo "Indice" in blu e grande


\title{Categorie di Classi}
\author{TaleWind}
\begin{document}
\maketitle
\tableofcontents
\newpage
\section{Introduzione}
Fabula Ultima offre molte classi al giocatore per poter vivere al meglio la sua esperienza, però spesso sorgono dei dubbi su quali classi siano più adatte per il nostro scopo sia esso il fare Danni, fornire supporto oppure semplicemente diventare un healer a tutti gli effetti.\\
Questo documento punta ad aiutare i player in difficoltà ad aprocciarsi alla creazione del personaggio di Fabula Ultima, Enjoy!

\section{Categorie di classi}
Nonostante esistano 28 classi è possibile, fortunatamente, dividerle in 3 categorie principali:
\begin{itemize}
    \item \textbf{Damage} : Classi il cui scopo principale è infliggere danni ingenti.
    \item \textbf{Support}: Classi il cui scopo principale è proteggere il party riducendo le prestazioni di combattimento nemiche.
    \item \textbf{Utility}: Classi il cui scopo principale è il miglioramento delle prestazioni, individuali e/o del party, dentro e fuori dal combattimento, senza agire esplicitamente sul danno.
\end{itemize}
Per comodità andremo a suddividere, almeno per questo momento, le classi in base al manuale di provenienza:
\subsection{Manuale Base}
\begin{itemize}
    \item \textbf{Damage} : Elementalista, Arcanista, Lama Oscura.
    \item \textbf{Support}: Guardiano, Spiritista, Oratore.
    \item \textbf{Utility}: Artefice, Canaglia, Chimerista, Entropista, Furia, Maestro D'armi, Sapiente, Tiratore, Viandante
\end{itemize}

\subsection{Manuale High Fantasy}

\begin{itemize}
    \item \textbf{Damage} : Cantore.
    \item \textbf{Support}: Simbolista.
    \item \textbf{Utility}: Comandante, Danzatore.
\end{itemize}

\subsection{Manuale Tecno Fantasy}

\begin{itemize}
    \item \textbf{Damage} : Mutante.
    \item \textbf{Utility}: Esper, Pilota.
\end{itemize}

\subsection{Manuale Natural Fantasy}

\begin{itemize}
    \item \textbf{Damage} : Invocatore.
    \item \textbf{Support}: Gourmet.
    \item \textbf{Utility}: Floralista, Mercante.
\end{itemize}


\subsection{Contenuto Bonus}

\begin{itemize}
    \item \textbf{Utility}: Asso delle Carte, Necromante.
\end{itemize}
Queste categorie sono volutamente generiche e pertanto non incapsulano perfettamente il funzionamento di 
tutte le classi, però consentono di effettuare una scrematura inziale in base al ruolo che il nostro personaggio 
avrà all'interno del party.
\section{Playstile}
La categorizzazione precedente fornisce una prima categorizzazione per le classi che si rivelerà utile per avere
una prima idea su cosa punta a fare la classe, però molte classi lo fanno in maniera diversa attraverso \textbf{Meccaniche}
diverse che molto spesso dettano la progressione del personaggio; Un esempio tipico è l'\textbf{Arcanista} che consente 
di vincolare gli \textbf{Arcana} per usarne i poteri.
\\
Pertanto catalogare il \textbf{Playstile} corrisponde ad evidenziare le unicità delle classi insieme alle azioni che utilizzano
maggiormente, insieme alla statistica principalmente interessata dalle abilità di classe.
\subsection{Manuale Base}
\begin{table}[h]
    \center
    \begin{tabular}{|c|c|c|c|c|c|c|}
      \hline
      \textbf{Classe} & \textbf{Benefici} & \textbf{Mana Usage} & \textbf{PI Usage}  & \textbf{Unicità} & \textbf{Status} & \textbf{Main Stat}\\
      \hline
      \centering Arcanista & \textcolor{blue}{PM} & \textcolor{green}{Alto} & \textcolor{red}{NO} & Arcanum & \textcolor{green}{SI} &\textcolor{blue}{VOL} \\
      \hline
      \centering Artefice & \textcolor{blue}{PI} & \textcolor{red}{NO} & \textcolor{green}{Alto} & Tecnologie & \textcolor{red}{NO} &\textcolor{blue}{None} \\
      \hline
      \centering Canaglia & \textcolor{blue}{PI} & \textcolor{green}{Basso} & \textcolor{red}{NO} & Ci Si Vede! & \textcolor{red}{NO} &\textcolor{blue}{DES, VOL} \\
      \hline
      \centering Chimerista & \textcolor{blue}{PM} & \textcolor{green}{Si} & \textcolor{red}{NO} & Magimimesi & \textcolor{green}{Variabile} &\textcolor{blue}{VOL} \\
      \hline
      \centering Elementalista & \textcolor{blue}{PM} & \textcolor{green}{Alto} & \textcolor{red}{NO} & None & \textcolor{green}{SI} &\textcolor{blue}{INT, VOL} \\
      \hline
      \centering Entropista & \textcolor{blue}{PM} & \textcolor{green}{Alto} & \textcolor{red}{NO} & Settebello & \textcolor{green}{SI} &\textcolor{blue}{INT, VOL} \\
      \hline
      \centering Furia & \textcolor{blue}{PV} & \textcolor{red}{NO} & \textcolor{red}{NO} & Provocazione & \textcolor{red}{NO} &\textcolor{blue}{VIG, VOL} \\
      \hline
      \centering Guardiano & \textcolor{blue}{PV} & \textcolor{red}{NO} & \textcolor{red}{NO} & Fortezza & \textcolor{red}{NO} &\textcolor{blue}{VIG} \\
      \hline
      \centering Lama Oscura & \textcolor{blue}{PV} & \textcolor{red}{NO} & \textcolor{red}{NO} & None & \textcolor{red}{NO} &\textcolor{blue}{VIG} \\
      \hline
      \centering Maestro D'armi & \textcolor{blue}{PV} & \textcolor{red}{NO} & \textcolor{red}{NO} & Maestria Armi Da Mischia & \textcolor{green}{SI} &\textcolor{blue}{None} \\
      \hline
      \centering Oratore & \textcolor{blue}{PM} & \textcolor{green}{Medio} & \textcolor{red}{NO} & Mi fido di Te & \textcolor{green}{SI} &\textcolor{blue}{INT, VOL} \\
      \hline
      \centering Sapiente & \textcolor{blue}{PM} & \textcolor{green}{Medio} & \textcolor{red}{NO} & Concentrazione & \textcolor{red}{NO} &\textcolor{blue}{INT} \\
      \hline
      \centering Spiritista & \textcolor{blue}{PM} & \textcolor{green}{Alto} & \textcolor{red}{NO} & Guarigione & \textcolor{green}{SI} &\textcolor{blue}{VOL} \\
      \hline
      \centering Tiratore & \textcolor{blue}{PV} & \textcolor{red}{NO} & \textcolor{red}{NO} & Maestria Armi a Distanza & \textcolor{green}{SI} &\textcolor{blue}{None} \\
      \hline
      \centering Viandante & \textcolor{blue}{PI} & \textcolor{red}{NO} & \textcolor{red}{NO} & Compagno Fedele & \textcolor{green}{Variabile} &\textcolor{blue}{None} \\
      \hline
    \end{tabular}
  \end{table}
\subsection{Atlanti Aggiuntivi}
  \begin{table}[h]
    \center
    \begin{tabular}{|c|c|c|c|c|c|c|}
      \hline
      \textbf{Classe} & \textbf{Benefici} & \textbf{Mana Usage} & \textbf{PI Usage}  & \textbf{Unicità} & \textbf{Status} & \textbf{Main Stat}\\
      \hline
      \centering Cantore  & \textcolor{blue}{PM}   & \textcolor{green}{Alto} & \textcolor{red}{NO} & Magicanto & \textcolor{green}{Si} &\textcolor{blue}{None} \\
      \hline
      \centering Comandante  & \textcolor{blue}{PV}   & \textcolor{red}{NO} & \textcolor{red}{NO} & Comandare il Party & \textcolor{red}{NO} &\textcolor{blue}{None} \\
      \hline
      \centering Danzatore & \textcolor{blue}{PV/PM}   & \textcolor{green}{SI} & \textcolor{green}{Basso} & Danze & \textcolor{green}{SI} &\textcolor{blue}{None} \\
      \hline
      \centering Simbolista& \textcolor{blue}{PI}   & \textcolor{green}{Basso} & \textcolor{green}{Alto} & Simbolismo & \textcolor{green}{SI} &\textcolor{blue}{None} \\
      \hline
      \centering Esper  & \textcolor{blue}{PM}   & \textcolor{green}{Alto} & \textcolor{red}{NO} & Doni Psichici & \textcolor{red}{NO} &\textcolor{blue}{VOL} \\
      \hline
      \centering Mutante  & \textcolor{blue}{PV}   & \textcolor{red}{NO} & \textcolor{red}{NO} & Colpi Senz'armi & \textcolor{red}{NO} &\textcolor{blue}{None} \\
      \hline
      \centering Pilota & \textcolor{blue}{PV}   & \textcolor{red}{NO} & \textcolor{red}{NO} & Veicolo Personale & \textcolor{red}{NO} &\textcolor{blue}{VIG} \\
      \hline
      \centering Floralista& \textcolor{blue}{PV/PM}   & \textcolor{green}{Medio} & \textcolor{red}{NO} & Magisemi & \textcolor{green}{Variabile} &\textcolor{blue}{None} \\
      \hline
      \centering Gourmet  & \textcolor{blue}{PI}   & \textcolor{green}{Medio} & \textcolor{green}{SI} & Delizie & \textcolor{red}{NO} &\textcolor{blue}{None} \\
      \hline
      \centering Invocatore  & \textcolor{blue}{PV/PM}   & \textcolor{green}{Basso} & \textcolor{red}{NO} & Invocazioni & \textcolor{green}{Variabile} &\textcolor{blue}{None} \\
      \hline
      \centering Mercante& \textcolor{blue}{PI}   & \textcolor{red}{NO} & \textcolor{red}{NO} & Punti Commercio & \textcolor{red}{NO} &\textcolor{blue}{None} \\
      \hline
      \centering Asso Delle Carte& \textcolor{blue}{PV/PM}   & \textcolor{green}{Alto} & \textcolor{red}{NO} & Carte & \textcolor{green}{Variabile} &\textcolor{blue}{None} \\
      \hline
      \centering Necromante& \textcolor{blue}{PV/PM}   & \textcolor{red}{NO} & \textcolor{red}{NO} & Punti Sepolcro & \textcolor{red}{NO} &\textcolor{blue}{None} \\
      \hline
    \end{tabular}
  \end{table}
\newpage
\subsection{Uso delle Tabelle}
Dalle tabelle di sopra si ricava un quadro generale abbastanza completo che può essere consultato per creare \\
delle \textbf{"Build"} in maniera più semplice e veloce. È sempre consigliato consultare il manuale di riferimento per quella classe.\\
Alcune attributi della tabella hanno un significato particolare che verrà spiegato di seguito:
\begin{itemize}
  \item \textbf{Benefici}: Sono i benefici gratuiti ottenuti dalla Classe, sono stati esclusi i benefici legati alle armi ed armature marziali perchè verranno considerati in seguito.
  \item \textbf{Mana Usage}: Rappresenta la richiesta di mana per utilizzare le abilità di classe ed eventuali incantesimi che la classe permette di ottenere, ad esempio il necromante non ha Mana Usage in quanto non consente di imparare incantesimi, sebbene potenzi quelli dell'entropista.
  \item \textbf{PI Usage}: Rappresenta la richiesta di Punti Inventario per utilizzare le abilità di classe, le azioni di Inventario non sono comprese.
  \item \textbf{Unicità}: Caratteristiche Polarizzanti della classe o punti di interesse per multiclassare, Curiosamente le classi con "None" sono quelle di categoria \textbf{Damage} \text{(è stato un caso)}.
  \item \textbf{Status}: Rappresenta la possibilità che la classe offre per infliggere status al nemico
  \item \textbf{Main Stat}: Rappresenta la/le caratteristiche \textbf{Richieste} per far funzionare al meglio la classe, le classi con la dicitura "None" o possono usarle tutte e 4, vedesi \textbf{Cantore}, oppure non hanno caratteristiche principali per il loro funzionamento.
\end{itemize}


\subsection{Armi ed Armature Marziali}
Per equipaggiare Armi ed Armature Marziali è necessario che i \textbf{Benefici Gratuiti} ce lo consentano; Ad esempio il guardiano oltre a +5 PV fornisce la possibilità di equipaggiare Armature e Scudi Marziali, di seguito verranno elencate le classi che forniscono questi benefici
\begin{itemize}
  \item \textbf{Armi da mischia Marziali} : Furia, Lama Oscura, Maestro D'armi, Comandante, Pilota
  \item \textbf{Armi a distanza Marziali} : Tiratore, Comandante, Pilota
  \item \textbf{Armature Marziali} : Guardiano, Furia, Lama Oscura
  \item \textbf{Scudi Marziali} : Guardiano, Maestro D'armi, Tiratore
  \item \textbf{Ritualismo} : Chimerista, Elementalista, Entropista, Ritualismo
\end{itemize}


\subsection{Rituali}
Similmente alle dotazioni marziali le discipline dei rituali possono essere ottenute solo tramite le abilità delle classi che come \textbf{Beneficio Gratuito} consentono di svolgere rituali di \textbf{Ritualismo} : Chimerista, Elementalista, Entropista, Ritualismo

\newpage











\end{document}

